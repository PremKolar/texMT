
Intuitively any translative motion of a vortex should stem from an asymmetry of forces as in an imperfectly balanced gyroscope wobbling around and translating across the table. The main effects that cause a quasi-geostrophic ocean eddy to translate laterally can easily be explained heuristically.
\begin{description}
\litem{Lateral Density Gradient}{speed_dens}
Consider a mean layer-thickness gradient $\dpr{h}{x}>0$ somewhere in the high northern latitudes and a geostrophic, positive density anomaly within that layer.
In other words a high-pressure vortex or an anti-cyclonic eddy with length scale $L\approx \mathrm{L_{R}}$. Hence a vorticity budget dominated by advection of
relative vorticity and vortex stretching. Consider a parcel of water adjacent to the eddy on its eastern flank. Due to the eddy's negative vorticity, the parcel
will be advected west into shallower layer-thickness where it will be squeezed vertically and acquire negative relative vorticity via term $C$ in \eqref{eq:vort7}.
Analogously a parcel of initially zero $\vec{\omega}$ on the eddy's western flank will be transported east, stretched and thereby acquire positive relative
vorticity. The result is that the eddy will be shoved south from both zonal flanks. Note that the rotational sense of the eddy is irrelevant here. The drift
direction is dictated by the sign of $f$. Hence eddies in the
northern hemisphere will be pushed along gradients with the shallower water always on their right and vice versa on the southern hemisphere.
\litem{\textit{Planetary Lift}}{speed_planlift}
Assume now that $\beta L$ be comparable or larger even than $f_{0}-\omega$ from the previous example. Then, all fluid adjacent to the eddy on its northern and southern flanks will be transported meridionally, thereby be tilted with respect to $\Omega$ and hence acquire relative vorticity to compensate. All fluid transported north will balance the increase in planetary vorticity with a decrease in relative vorticity and vice versa for fluid transported south. This is again independent of the eddy's sense and in this case also independent of hemisphere since $\dpr{f}{y}=\beta>0$ for all latitudes. The result is that small negative vortices to the northern and small positive vortices to the southern flank of eddies will push them west.
\litem{Eddy-Internal $\beta$-Effect}{speed_beta}
In the later case clearly particles within the vortex undergo a change in planetary vorticity as well. Or from a different point of view, since $U \sim \grad p/f  $, and noting that the pressure gradient is the driving force here and hence fix at first approximation, particles drifting north will decelerate and those drifting south will accelerate. In order to maintain mass continuity, the center of volume will be shifted west for an anti-cyclone and east for a cyclone. Another way to look at it is to note that the only way for the discrepancy in Coriolis acceleration north and south, whilst maintaining constant eddy-relative particle speed, is to superimpose a zonal drift velocity so that net particle velocities achieve symmetric Coriolis acceleration.
\end{description}

%{\color{red} equations to follow \cite{Cushman-Roisin1990} \cite{VanLeeuwen2007}}
\newpage
