% ######################################################################
% OTHER
% ######################################################################
\let\oldeqref\eqref
\let\oldhat\hat
\newcommand{\rnm}[1]{\romannumeral #1}
\newcommand{\Rmn}[1]{\Roman #1}
\newcommand{\footder}[1]{\footnote{See section \ref{#1} for derivation.}}
\definecolor{colboxcolor}{HTML}{9ECBFF}
\definecolor{legendcol}{HTML}{7BBCBD}
\newcommand{\HRule}{\rule{\linewidth}{0.5mm}} % Defines a new command for the horizontal lines, change thickness here
\newcommand*{\eg}{e.g.\@\xspace}
\newcommand*{\ie}{i.e.\@\xspace}
\newcommand*{\Eg}{E.g.\@\xspace}
\newcommand*{\Ie}{I.e.\@\xspace}

%\makeatletter
\newcommand*{\etc}{\@ifnextchar{.}{etc}{etc.\@\xspace}}
%\makeatletter
\newcommand{\litem}[2]{\def\@itemlabel{\textbf{#1}}\item\def\@currentlabel{\textbf{#1}}\label{#2}}
%\makeatother
\newcommand{\todoil}[1]{\todo[color=red]{#1}}

\newcommand{\todoilc}[2]{\todo[color=#1]{#2}}

\newcommand{\derref}[1]{\footnote{see\fcolorbox{white}{gray!20}{Derivation \ref{#1}}}}

\newcommand{\derrefs}[1]{\footnote{see\fcolorbox{white}{gray!20}{Derivations \cref{#1}}}}

\renewcommand{\dateseparator}{-}

\newcommand\acknowledgements[1]
{
	\btypeout{Acknowledgements}
	\thispagestyle{plain}
	\begin{center}{\large{\textit{Acknowledgements}}\par}\end{center}
	{\normalsize #1}
	\vfil\vfil\null
}

\newcommand\Declaration[1]
{
	\btypeout{Declaration of Authorship}
	\thispagestyle{plain}
	\null\vfil
	%\vskip 60\p@
	\begin{center}{\huge\bf Declaration of Authorship\par}\end{center}
	%\vskip 60\p@
	{\normalsize #1}
	\vfil\vfil\null
	%\cleardoublepage
}

%\fancypagestyle{fancybook}{%
    %\fancyhf{}%
    %% Note the ## here. It's required because \fancypagestyle is making a macro (\ps@fancybook).
    %% If we just wrote #1, TeX would think that it's the argument to \ps@fancybook, but
    %% \ps@fancybook doesn't take any arguments, so TeX would complain with an error message.
    %% You are not expected to understand this.
    %\renewcommand*{\sectionmark}[1]{ \markright{\thesection\ ##1} }%
    %\renewcommand*{\chaptermark}[1]{ \markboth{\chaptername\ \thechapter: ##1}{} }%
    %% Increase the length of the header such that the folios
    %% (typography jargon for page numbers) move into the margin
    %\fancyhfoffset[LE]{6mm}% slightly less than 0.25in
    %\fancyhfoffset[RO]{6mm}%
    %% Put some space and a vertical bar between the folio and the rest of the header
    %\fancyhead[LE]{\thepage\hskip3mm\vrule\hskip3mm\leftmark}%
    %\fancyhead[RO]{\rightmark\hskip3mm\vrule\hskip3mm\thepage}%
%}
% ######################################################################
% ENVIRONTMENTS
% ######################################################################

\newenvironment{colbox}[1]
{
	\def\FrameCommand{\colorbox{colboxcolor}}%
	\MakeFramed{\advance\hsize-\width \FrameRestore\cornersize{0.9}}
	\begin{scriptsize}
	\section{#1}
}
{
    \end{scriptsize}
	\endMakeFramed
}
% -----------------------
\newenvironment{colbox2}[1]
{
	\def\FrameCommand{\colorbox{legendcol}}%
	\MakeFramed{\advance\hsize-\width \FrameRestore}
	\begin{footnotesize}
}
{
\end{footnotesize}
	\endMakeFramed
}
% -----------------------
\newenvironment{abstract}%
{
    \cleardoublepage\thispagestyle{empty}\null\vfill\begin{center}%
    \bfseries\abstractname\end{center}
}
{
	\vfill\null
}
%  #####################################################################
% STYLES
% ######################################################################
 \mdfdefinestyle{definition}
 {linewidth=1,
 backgroundcolor=yellow!40,
 outerlinecolor=blue!70!black,
 frametitlebackgroundcolor=gray!20,
 % frametitlerule=true,
 innertopmargin=\topskip,}
 \mdtheorem[
 style=definition]{definition}{Definition}
 \mdfdefinestyle{codepiece}
 {linewidth=15pt,
 linecolor=black,
 %frametitlerule=true,
 backgroundcolor=red!1,
 outerlinecolor=black,
 frametitlebackgroundcolor=blue!10,
 innertopmargin=\topskip,}
 \mdtheorem[
 style=codepiece]{codepiece}{Code}[chapter]
 \mdfdefinestyle{function}
 {%linewidth=15pt,
 %linecolor=black,
 %frametitlerule=true,
 %backgroundcolor=red!1,
 %outerlinecolor=black,
 %frametitlebackgroundcolor=blue!10,
 innertopmargin=-.5cm,}
 \mdtheorem[
 style=function]{function}{sub-routine}[section]
 \mdfdefinestyle{derivation}
 {linecolor=red,
 outerlinewidth=2,
 leftmargin=40mm,
 rightmargin=40mm,
 linewidth=2,
 frametitlerule=true,
 frametitlebackgroundcolor=gray!20,
 innertopmargin=\topskip,roundcorner=10pt,}
 \mdtheorem[
 style=derivation]{derivation}{Derivation}
 \mdfdefinestyle{eddy}
 {linewidth=.5,
 % align=center,
 % backgroundcolor=yellow!40,
 outerlinecolor=black,
 frametitlebackgroundcolor=gray!20,
 % frametitlerule=true,
 innertopmargin=\topskip,}
 \mdtheorem[
 style=eddy]{eddy}{Vortex}[chapter]
 \mdtheorem[
 style=eddy]{turbu}{Turbulence}[chapter]
%  #####################################################################
% MATH
% ######################################################################
\renewcommand{\vec}[1]{\mathbf{\bm{#1}}}
\newcommand{\ten}[1]{\mathbb{\bm{#1}}}
\newcommand{\tent}[1]{\mathbb{\bm{#1}}^{\mathsmaller T}}
\renewcommand{\hat}[1]{\oldhat{\mathbf{#1}}}
\renewcommand{\l}{\left(}
 \renewcommand{\r}{\right)}
\newcommand{\pr}{\partial}
\newcommand{\grad}{\vec{\nabla}}
\newcommand{\arccosh}{\mathrm{arccosh}}
\newcommand{\lapl}{\vec{\triangle}}
\newcommand{\curl}{\grad \times}
\newcommand{\gradt}{ \breve{\grad}}
\renewcommand{\div}{\grad \cdot}
\newcommand{\norm}[1]{\left\lVert#1\right\rVert}
\renewcommand{\eqref}[1]{equation \oldeqref{#1}}
\newcommand{\eqsref}[1]{equations \oldeqref{#1}}
\newcommand{\Eqref}[1]{Equation \oldeqref{#1}}
\newcommand{\Eqsref}[1]{Equations \oldeqref{#1}}
\newcommand{\dpr}[2]{\frac{\partial#1}{\partial#2}}
\newcommand{\Dpr}[2]{\frac{D#1}{D#2}}
\newcommand{\Dprs}[2]{\frac{D^{\star}#1}{D#2}}
\newcommand{\advec}[1]{\vec{u} \cdot \vec{\nabla} #1}
\newcommand{\unit}[1]{\left[#1\right]}
\newcommand{\unitvec}[1]{\hat{\vec{#1}}}
\newcommand{\tvec}[1]{\underaccent{\neg}{\vec{#1}}}
\newcommand{\tsca}[1]{\underaccent{\neg}{#1}}
\newcommand{\order}[1]{\mathcal{O}\left( 10^{#1}\right)}
\newcommand{\sign}[1]{\mathrm{sgn}\left(#1\right)}
\newcommand{\abs}[1]{\left|#1\right|}
\DeclareMathOperator{\tr}{Tr}


%% legend stuff
\newcommand{\Bu}[0]{\mathrm{\hyperref[def:Bu]{Bu}}\;}
\newcommand{\Ro}[0]{\mathrm{\hyperref[def:Ro]{Ro}}\;}
\newcommand{\Rh}[0]{\mathrm{\hyperref[def:Rh]{R_{\beta}}}\;}
\newcommand{\Lr}[0]{\mathrm{\hyperref[def:Lr]{L_{R}}}\;}
\newcommand{\Lb}[0]{\mathrm{\hyperref[def:Lb]{L_{\beta}}}\;}
\newcommand{\h}[0]{\hyperref[def:h]{h}}
\newcommand{\B}[0]{\hyperref[def:B]{ \vec{B}}}
\newcommand{\Ek}[0]{\hyperref[def:Ek]{E_k}}
\newcommand{\Em}[0]{\hyperref[def:Em]{E_m}}
\newcommand{\enstro}[0]{\hyperref[def:enstro]{\varepsilon}}
\newcommand{\f}[0]{\mathit{\hyperref[def:f]{f}}\;}
\newcommand{\dfdy}[0]{\mathrm{\hyperref[def:beta]{\beta}}\;}
\newcommand{\g}[0]{\mathit{\hyperref[def:g]{g}}\;}
\newcommand{\okubo}[0]{\mathrm{\hyperref[def:okubo]{O_w}}\;}
\newcommand{\SSH}[0]{\hyperref[def:SSH]{SSH}\;}
\newcommand{\IQ}[0]{\mathrm{\hyperref[def:IQ]{IQ}}\;}
\newcommand{\rG}[0]{\hyperref[def:rG]{\mathfrak{r}}\;}
\newcommand{\PtoA}[0]{\hyperref[TODO]{POP2Aviso}\@\xspace}
\newcommand{\MI}[0]{\hyperref[TODO]{MI}\@\xspace}
\newcommand{\MII}[0]{\hyperref[TODO]{MII}\@\xspace}


\newcommand{\Enstro}{\mathcal{E}}
\newcommand{\inta}[1]{ \int_A #1 \; \mathrm{d}A}
\newcommand{\intm}[1]{ \frac{1}{A} \int_A #1 \; \mathrm{d}A}
\newcommand{\dint}[1]{ \; \mathrm{d}#1}
%\newcommand{\d}[1]{\mathrm{d}#1}
\newcommand{\expp}[1]{\mathrm{e}^{#1}}
\newcommand{\INT}[4]{\int_{#1}^{#2} #3 \; \mathrm{d}#4}
\newcommand{\intms}[1]{ \left< #1 \right>}
\renewcommand{\deg}[1]{#1^\circ}
\newcommand{\decom}[1]{\overline{#1} + #1'}
\newcommand{\inbr}[1]{\left( #1 \right)}
\newcommand{\ol}[1]{\overline{#1}}
\newcommand{\timesES}[2]{\epsilon_{jki} #1_j #2_k \unitvec{e}_i}
\newcommand{\oh}[0]{\frac{1}{2}}
% ######################################################################
% VARIABLES
% ######################################################################
\newcommand*{\supervisor}[1]{\def\supname{#1}}
\newcommand*{\thesistitle}[1]{\def\ttitle{#1}}
\newcommand*{\examiner}[1]{\def\examname{#1}}
\newcommand*{\degree}[1]{\def\degreename{#1}}
\newcommand*{\authors}[1]{\def\authornames{#1}}
\newcommand*{\addresses}[1]{\def\addressnames{#1}}
\newcommand*{\university}[1]{\def\univname{#1}}
\newcommand*{\UNIVERSITY}[1]{\def\UNIVNAME{#1}}
\newcommand*{\department}[1]{\def\deptname{#1}}
\newcommand*{\DEPARTMENT}[1]{\def\DEPTNAME{#1}}
\newcommand*{\group}[1]{\def\groupname{#1}}
\newcommand*{\GROUP}[1]{\def\GROUPNAME{#1}}
\newcommand*{\faculty}[1]{\def\facname{#1}}
\newcommand*{\FACULTY}[1]{\def\FACNAME{#1}}
\newcommand*{\subject}[1]{\def\subjectname{#1}}
\newcommand*{\keywords}[1]{\def\keywordnames{#1}}
\newcommand\btypeout[1]{\bhrule\typeout{\space #1}\bhrule}
\newcommand\bhrule{\typeout{------------------------------------------------------------------------------}}
% ######################################################################
