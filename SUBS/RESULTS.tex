\defcitealias{Chelton2011}{them}

The short time-frame and limited computational resources allowed for only a few complete global runs over the available data.
It was therefor critical to carefully choose which method/parameters to use in order to maximize the deducible insight from the results.
For best comparability of the results with each other it was decided to agree on one complete set of parameters as a basis, which would then be altered at key parameters.
The first run is an attempt to reproduce the results from \citet{Chelton2011}, by setting the algorithm to be the most similar to the algorithm described by \citetalias{Chelton2011}.  The SSH-data for this run is therefor that of the Aviso product.
This method will be called \MI.\\
 The second run is equivalent, except that this time the alternative $\IQ$-based shape filtering method from TODO ref and the slightly different tracking-filter as described in TODO ref are used. This setting will be called \MII. \MII is then fed with 7-day time-step POP data as well.\\
 To investigate what role space-resolution plays, the POP data was remapped to that of the Aviso data and fed to the \MI method. Finally, to investigate the effects of resolution in time, an \MII/3-day-time-step run over POP data was executed.  \\
  Start and end dates were fix for all runs as the intersection of availability of both data sets (see table \ref{tab:fixparams} for details).


\begin{table} 
	\begin{tabularx}{\textwidth}{|X|X|}
	\hline
	time frame &  \displaydate{runsStart} till \displaydate{runsEnd}\\
	\hline
	scope & full globe ($80S:80N \;\; 180W:180E$) \\
	\hline
	AVISO geometry &   $641 x 1440$ true Mercator \\
	\hline
	POP   geometry &   $2400 x 3600$ \\
	\hline
	contour step   &   \contourstep \\
	\hline
	\end{tabularx}
	\begin{tabularx}{\textwidth}{|X|X|}
	\hline
	\textbf{thresholds} &  [all SI]  \\
	\hline
	max $\sigma/\Lr$ & \threshmaxRadiusOverRossbyL \\
	\hline
	min $\Lr$ TODO & \threshminRossbyRadius \\
	\hline
	min $\IQ$ & \threshshapeiq \\
	\hline
	min data points of an eddy & \threshcornersmin \\
	\hline
	max(abs(rossby phase speed)) TODO & \threshphase \\
	\hline
	\end{tabularx}	
	\caption{Fix parameters for all runs.}
\label{tab:fixparams}
\end{table}
\todoil{lookup term used for krummes grid}


\begin{table} 
	\begin{tabularx}{\textwidth}{|X||X|X|X|}
	\hline
	\textbf{SSH-data} & POP & Aviso  & \PtoA  \\
	\hline
	\textbf{method}   & \MI  &  \MII  &    \\
	\hline
	\textbf{time-step}   & 7 &  3  &  \\
	\hline
	\end{tabularx}
	\caption{Variable parameters.}
\label{tab:varparams}
\end{table}


\section{Method \MI}
The concepts used in this method are mostly based on the description of the algorithm described by \citep{Chelton2011} and all parameters are set accordingly. Basically \MI is a modification of \MII (which was completed first), with the aim to try to recreate the results from \citep{Chelton2011}.
It differs from \MII in the following:
\begin{itemize}
	\item \textbf{detection}
As mentioned in TODO ref, the approach by \citet{Chelton2011} is to avoid overly elongated objects by demanding:
\begin{itemize}
	\item high latitudes\\
	The maximum distance between any vertices of the contour must not be larger than $400km$ for $\abs{\phi}>\deg{25}$.
	\item low latitudes\\
The $400km$-threshold increases linearily towards the equator to $1200km$.
\end{itemize}
	\item \textbf{tracking}
The other minor differerence to \MII is in the way the tracking algorithm flags eddy-pairs between time-steps as sufficiently similar to be considered successful tracking-candidates (see TODO ref).
In this method an eddy B from time-step $k+1$ is considered as a potential manifestations of an eddy A from time-step $k$ as long as both - the ratio of amplitudes (with regard to the mean of SSH within the found contour) and the ratio of areas (interpolated versions as discussed in TODO ref) fall within a lower and and an upper bound.
\end{itemize}


\section{Method \MII}
Even though, in its core, directly inspired by \citep{Chelton2011}, this method differs from \MII and thus from the description by \citeauthor{Chelton2011} mainly in the way the shape of a found contour is deemed sufficiently eddy-like.
\begin{itemize}
	\item \textbf{detection}
$\IQ$
	\item \textbf{tracking}
$\max{\left(\left[\exp{\abs{\log{R_{\alpha}}}}\; ; \;\; \exp{\abs{\log{R_{\sigma}}}} \right]\right)} $
\end{itemize}
